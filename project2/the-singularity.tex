%%%%%%%%%%%%%%%%%%%%%%%%%%%%%%%%%%%%%%%%%
% Simple Sectioned Essay Template
% LaTeX Template
%
% This template has been downloaded from:
% http://www.latextemplates.com
%
% Note:
% The \lipsum[#] commands throughout this template generate dummy text
% to fill the template out. These commands should all be removed when 
% writing essay content.
%
%%%%%%%%%%%%%%%%%%%%%%%%%%%%%%%%%%%%%%%%%

%----------------------------------------------------------------------------------------
%	PACKAGES AND OTHER DOCUMENT CONFIGURATIONS
%----------------------------------------------------------------------------------------

\documentclass[12pt]{article} % Default font size is 12pt, it can be changed here

\usepackage{geometry} % Required to change the page size to A4
\geometry{a4paper} % Set the page size to be A4 as opposed to the default US Letter

\usepackage{graphicx} % Required for including pictures

\usepackage{float} % Allows putting an [H] in \begin{figure} to specify the exact location of the figure
\usepackage{wrapfig} % Allows in-line images such as the example fish picture

\usepackage{url} % Allows for urls to be formatted decently well

\usepackage{natbib}

% \usepackage{lipsum} % Used for inserting dummy 'Lorem ipsum' text into the template

\linespread{1.2} % Line spacing

%\setlength\parindent{0pt} % Uncomment to remove all indentation from paragraphs

\graphicspath{{img/}} % Specifies the directory where pictures are stored

\begin{document}

%----------------------------------------------------------------------------------------
%	TITLE PAGE
%----------------------------------------------------------------------------------------

\begin{titlepage}

\newcommand{\HRule}{\rule{\linewidth}{0.5mm}} % Defines a new command for the horizontal lines, change thickness here

\center % Center everything on the page

\textsc{\LARGE Wheaton College (MA)}\\[1.5cm] % Name of your university/college
\textsc{\Large COMP-401}\\[0.5cm] % Major heading such as course name
\textsc{\large Senior Seminar}\\[0.5cm] % Minor heading such as course title

\HRule \\[0.4cm]
{ \huge \bfseries Why AGI Is Not Necessarily Imminent}\\[0.4cm] % Title of your document
\HRule \\[1.5cm]

\begin{minipage}{0.4\textwidth}
\begin{flushleft} \large
\emph{Author:}\\
Bryan \textsc{Jensen} % Your name
\end{flushleft}
\end{minipage}
~
\begin{minipage}{0.4\textwidth}
\begin{flushright} \large
\emph{Professor:} \\
Tom \textsc{Armstrong} % Supervisor's Name
\end{flushright}
\end{minipage}\\[4cm]

{\large \today}\\[3cm] % Date, change the \today to a set date if you want to be precise

%\includegraphics{Logo}\\[1cm] % Include a department/university logo - this will require the graphicx package

\vfill % Fill the rest of the page with whitespace

\end{titlepage}

%----------------------------------------------------------------------------------------
%	TABLE OF CONTENTS
%----------------------------------------------------------------------------------------

\tableofcontents % Include a table of contents

\newpage % Begins the essay on a new page instead of on the same page as the table of contents 


% \begin{figure}[H] % Example image
% \center{\includegraphics[width=0.5\linewidth]{ontheedge}}
% \caption{Example image.}
% \label{fig:speciation}
% \end{figure}



% \begin{wrapfigure}{l}{0.4\textwidth} % Inline image example
%   \begin{center}
%     \includegraphics[width=0.38\textwidth]{logisticcurve}
%   \end{center}
%   \caption{Fish}
% \end{wrapfigure}



% \begin{description} % Numbered list example

% \item[First] \hfill \\
% % \lipsum[9] % Dummy text

% \item[Second] \hfill \\
% % \lipsum[10] % Dummy text

% \item[Third] \hfill \\
% % \lipsum[11] % Dummy text

% \end{description} 



% Example citation \cite{Figueredo:2009dg}.



% \bibitem[Figueredo and Wolf, 2009]{Figueredo:2009dg}
% Figueredo, A.~J. and Wolf, P. S.~A. (2009).
% \newblock Assortative pairing and life history strategy - a cross-cultural
%   study.
% \newblock {\em Human Nature}, 20:317--330.

%----------------------------------------------------------------------------------------
%	INTRODUCTION
%----------------------------------------------------------------------------------------

\section{Introduction} % Major section

Artificial Intelligence (AI) has been in the news quite a lot lately, in a different way than ever before. Big names in technology and science have been voicing their concerns in interviews and over social media about possible imminent dangers of AI. The people voicing these concerns are public figures in science and technology such as Elon Musk\cite{muskinterview} and Stephen Hawking\cite{hawkinginterview}. And, despite them not being experts specifically in AI, they are considered among the smartest in the world.

They are concerned about the possibility of developing an Artificial Intelligence that exceeds human intelligence. Since such an AI could be used to aid in the process of improving itself, we would soon see exponential growth in its abilities, building towards an event sometimes called ``The Singularity''\cite{wbw}. This amount of power could rival the nuclear bomb, and since it is itself intelligent, we have no guarantee of our ability to control it.

The main impetus for the recent levels of concern are new estimates for the proximity of the singularity. Experts have started placing the most likely timing of this event to be within the next few decades, and the vast majority of AI experts agree that it will certainly happen within the century\cite{wbw}. The term \textit{singularity} is, in general, used to refer to the moment where a function goes to infinity, most commonly used for the density of matter at the center of a black hole. In the field of AI, it is used to refer to the moment when progress, specifically in technology, starts to accelerate to the extent where, to humans, it may as well be infinitely fast.

Part of what will feed into AI's ability to produce this exponential growth is the observed exponential growth in computing technology that has taken place over the past few decades. This is most often quoted as ``Moore's Law,'' which states that the density of transistors on consumer-grade processors would double every two years. There are similar laws that deal with the actual speed of the processors, the computing power of those processors, and more; all values which are closely related but not identical. Note, the original Moore's Law time frame was 24 months, and lately (since 2013)\cite{mooresslowing} seems to be increasing to beyond 36. Even so, it is a large increase in computing power happening every year, and it could still allow for huge advances in what's technically possible.

%----------------------------------------------------------------------------------------
%	Artificial Intelligence
%----------------------------------------------------------------------------------------

\section{Artificial Intelligence} % Major section

%------------------------------------------------

\subsection{AI in Popular Culture}

By far the most commonly known form of AI is the kind of AI that finds its way into movies, from a larger series such as The Matrix to a newer entry such as Her. This tends to be, as you would expect, a romanticized version of AI, more along the lines of what was originally envisioned as a rival to human intelligence. Also, in the movies most often the AI is embodied in a robot, but as it happens the physical machine that makes the ``robot'' ends up just being a container for the AI, just as our bodies could be considered containers for our brains.

%------------------------------------------------

\subsection{AI in the ``Real World''}

AI research currently tends more towards the simplification of a given problem, rather than generalization of the technique. What this means is that they take a problem that is viewed as ``hard,'' such as the game of chess, and break it down into solvable, computable chunks. Often this ends up reducing the given problem into a tree, graph or other searchable space, against which we have many well developed tools and algorithms we can leverage.

Another form is a rule-based structure that matches situations with solutions for that situation. This is popular in video gaming, where events can trigger reactions in the AI, intended to address that event to the furthering the AI's goals. For example, a ``bot'' in a computer game may, when presented with a human player with very little remaining health, may employ a tactic to play aggressively and kill the human player, thus completing the corresponding goal pre-programmed into the ``bot.''

%------------------------------------------------

\subsection{What I consider Artificial \textit{Intelligence}}

My definition of AI stresses the \textit{intelligence} half of the concept, defining it as a process that learns - this encompasses a suite of algorithms referred to as ``genetic algorithms.'' The central concept behind these algorithms is to mimic evolution and thereby learn how to solve a problem starting from complete ignorance and eventually, in theory, becoming a master. The key to these algorithms is that they themselves learn how to solve the problem, and do not rely any outside knowledge.

%------------------------------------------------

\subsection{Real Examples of AI}

There are numerous examples of AI in the world at the moment. For a short while, the most popular of these was DeepBlue, the computer built solely to win chess against the world grand master, a task which it accomplished\cite{deepblue}. Another famous example is Watson, designed by IBM to be capable of beating the world's best players at the TV Show Game \textit{Jeopardy!}\cite{watson}, which it managed to do handily to a large audience\cite{watsonwins}. A final example, one less well known but likely recognizable, is an attempt by Google to design a system to recognize text in pictures, a system which was actually capable of breaking the spambot-filter systems throughout the internet, known as ``captcha'' or ``reCaptcha''. These are all examples of ``Artificial Narrow Intelligence,'' by far the most common type of AI in the world today.

%----------------------------------------------------------------------------------------
%   Progression of AI
%----------------------------------------------------------------------------------------

\section{Progression of AI}

%------------------------------------------------

\subsection{Artificial Narrow Intelligence} % Sub-section

ANI covers \textit{everything} in the world of practical AI in the world today. Also called \textit{Weak AI}\cite{wikiani}, it does one job, perhaps exceptionally well, but it does nothing else. For example, Watson is an impressive feat of programming and technology, but it only does what it does. Any attempt to use it to play a game of chess, and it will have no idea what to do.

%------------------------------------------------

\subsection{Artificial General Intelligence} % Sub-section

AGI, often called \textit{Strong AI}\cite{wikiagi} covers the entire concept of a form of AI that can compete with a human in every respect, from chess to talking to figuring out how to make a cup of coffee. This has been the goal of AI research since its inception, and is expected to be the invention that sparks the singularity.

%------------------------------------------------

\subsection{Artificial Super Intelligence} % Sub-section

An AI in this category belongs to a realm of intelligence incomprehensible to humans\cite{wbw}. An ASI entity trying to explain what it knows to us would be analogous to humans trying to explain relativity to a chimp. There is simply no understanding on the behalf of the recipient. It is expected that once we achieve AGI, that entity can be used to improve itself repeatedly, and it won't be long before we find ourselves in the presence of an ASI.


%----------------------------------------------------------------------------------------
%   Moore's Law
%----------------------------------------------------------------------------------------

\section{Moore's Law}

With how scary that eventuality is, it is no surprise that it is what gets the majority of the focus when talking about advances in technology and artificial intelligence. However, the vast majority of talk and predictions about the upcoming singularity base its inevitability on the steady and massive increases in technological power at our disposal\cite{wbw2}. The thinking is, if we develop a powerful enough computer, then our algorithm doesn't have to be efficient to be smart, and by increasing that power we can eventually manufacture a system smarter than humans.

The problem with this scenario is its reliance on Moore's Law. Moore's Law has held true for half a century of computational advances, so why would it come into question now? The issue is that, while it has held for this long, it has slowed over time and there is no reason to believe it will provide the driving power for the creation of AGI.

%------------------------------------------------

\subsection{What is Moore's Law?} % Sub-section

Moore's Law, at its essence, dictates that computing power increases at an exponential rate with regards to time. Its official form states that ``the number of transistors per square inch on consumer-grade processors will double roughly every two years.''

%------------------------------------------------

\subsection{What's Wrong with Moore's Law?} % Sub-section

The law was originally stated in 1965, and since then the law has retained its truthfulness - for the most part. Part of the reason this has been true is that the law became a self-fulfilling prophecy; it became the industry goal, and hence spawned a second, sister Moore's Law (also known as Rock's Law): the cost of an Research \& Development lab would also increase exponentially over time. With money being a finite resource, it is understandable why both of these could not continue forever\cite{memristor}.

However, all this being said, Moore's Law has encountered and overcome hurdles in the past\cite{hurdle}\cite{speeds}. I don't have the space to enumerate them in this paper, but one such is the shift from single-core to multi-core around 2006.

%------------------------------------------------

\subsection{AGI or ASI can solve it!} % Sub-section

Assuming we can continue the march of Moore's Law long enough to develop full AGI, the common thinking is that the new entity will be smart enough to help develop its own advances in computing power\cite{wbw}, and solve that issue while simultaneously improving itself in other ways. There is utter faith that if we manage to develop full ASI, then it will be able to solve all such problems, as its cognitive abilities will be just that far beyond ours - we may not understand it, but we believe in it.

%----------------------------------------------------------------------------------------
%   Inevitable Progress
%----------------------------------------------------------------------------------------

\section{Inevitable Progress} % Major section

% Inline image
\begin{wrapfigure}{l}{0.50\textwidth}
  \begin{center}
    \includegraphics[width=0.48\textwidth]{curvyexponential.png}
  \end{center}
  \caption{Inexorable Progress}
\end{wrapfigure}

Many people quote a certain fact as the driving reason why we must reach AGI, and therefore ASI: human progress is, historically, exponential\cite{wbw}. It may seem to slow for periods at a time, but eventually we make such leaps and bounds that society quickly becomes unrecognizable. Society is vastly different in 2015 from that of 1800, because of the progress we've made in the past 200 years. But we have to go more than 200 years before 1800 to reach the same level of change.

One main issue with this line of argument is, despite our newfound ability to predict that progress \textit{will} happen, we still seem to be incapable of predicting what form it will take. It has been decades since we were first convinced that flying cars were to be the method of transportation in the future, and still we drive ours along the same old paved roads.

Hence my argument becomes not that the human race \textit{won't} progress, but rather that we simply don't know what that innovation will look like. For certain we cannot know that ASI is the form our next big leap will take.

%----------------------------------------------------------------------------------------
%	Conclusion
%----------------------------------------------------------------------------------------

\section{Conclusion} % Major section

I will not say that AGI and ASI are impossible parts of our (very near) future. I wish to emphasize that, with exponential growth resulting in the greatest and quickest advancements that the human race has ever seen, we are in a time where we are unable to see even a decade into the future; those 10 years may be more different from today than 2015 is from 1915. However, it is also possible that 2025 may come and go without much change from our current situation.

I will not, however, believe claims stating that AGI is imminent, when those claims are based almost solely on Moore's Law. While Moore's Law is projected to hold out for another possible two decades, even today its progress is slowing, and it cannot be relied upon as the driving force behind \textit{true} artificial intelligence. I will believe that within my lifetime we will see many more changes that dramatically alter society, quite likely more drastically than in the past \textit{millennium}. But we have no way of knowing what form those changes will take, or even if they will be instigated by technology as we currently understand it.

Artificial Narrow Intelligence is a powerful force in our world today. Artificial General Intelligence has the power to easily change our world. Artificial Super Intelligence has the power to easily end our world. But none of it is guaranteed to happen, at least not as the experts seem convinced is imminent.

%----------------------------------------------------------------------------------------
%	BIBLIOGRAPHY
%----------------------------------------------------------------------------------------

% \begin{thebibliography}{99} % Bibliography - this is intentionally simple in this template

% \bibitem[Figueredo and Wolf, 2009]{Figueredo:2009dg}
% Figueredo, A.~J. and Wolf, P. S.~A. (2009).
% \newblock Assortative pairing and life history strategy - a cross-cultural
%   study.
% \newblock {\em Human Nature}, 20:317--330.

% \bibitem{Simpson} H. Simpson, \emph{Proof of the Riemann
% Hypothesis},  preprint (2003), available at 
% \url{http://www.math.drofnats.edu/riemann.ps}.
 

\bibliographystyle{plain}
\bibliography{the-singularity}

% \end{thebibliography}

%----------------------------------------------------------------------------------------

\end{document}